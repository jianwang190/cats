\documentclass[11pt]{report}

\usepackage[T1]{fontenc}
\usepackage[polish]{babel}
\usepackage[utf8]{inputenc}
\usepackage{lmodern}
\usepackage[pdftex]{graphicx}
\usepackage{url}
\selectlanguage{polish}
\usepackage{amsmath}
\usepackage[top=3cm, bottom=3cm, left=3cm, right=3cm]{geometry}
\usepackage{algorithm,algorithmic}
\title{Tytuł naszej pracy inżynierskiej}
\author{Katarzyna Śmietanka}
\begin{document}

%
% przeczytaj: http://en.wikibooks.org/wiki/LaTeX/Document_Structure
%

\maketitle
\tableofcontents

\chapter{Wstęp}

\section{Kontekst zagadnienia}

\par Problem układania planu zajęć dotyczy wielu instytucji głównie związanych ze szkolnictwem, od szkoły podstawowej aż to szkół wyższych. Zadanie sprowadza się do przygotowania graficznego rozkładu zajęć np. dla poszczególnych klas - uczniów uczęszczających do danej klasy, planu zajęć dla każdego z uczących nauczycieli oraz rozkładu zajęć dla każdej sali. Układanie planów zajęć nawet dla niewielkiej szkoły wymaga dużego wysiłku oraz nakładu czasowego. Problem staje jeszcze bardziej skomplikowany dla szkół ponadgimnazjalnych, do których uczęszcza więcej uczniów, tym samym jest więcej klas, nauczycieli oraz realizowanych przedmiotów. Ponadto zagadnienie ułożenia planu zajęć dla szkół wyższych, w których studenci mogą sami wybierać przedmioty w znacznym stopniu komplikuje te zagadnienie. Wynika to z tego, że aby dany plan zajęć był właściwy musi on spełniać podstawowe kryteria: dwa różne zajęcia na które jest zapisany minimum jeden student nie mogą się odbywać w tym samym czasie, dwa różne zajęcia w tym samym czasie nie mogą odbywać się w tej samej sali oraz nauczyciel w danym momencie może nauczać tylko jednego przedmiotu. Rozszerzeń tych ograniczeń jest wiele, a uwzględnienie każdego z nich, często konieczne do stworzenia realnego planu zajęć jest niezbędne. Wprowadzanie kolejnych ograniczeń znacznie utrudnia problem, który i tak jest stosunkowo skomplikowany.
\par Pomysł tematu pracy inżynierskiej wyniknął z przyglądania się na przestrzeni naszej edukacji planom zajęć które często odbiegały od idealnego według naszej opini. Plan zajęć często zawierał sporo przerw między obowiązkowymi zajęciami, zajęcia zaczynały się bardzo wcześnie lub kończyły bardzo późno, co w znacznym stopniu utrudniało powrót do domu bądź też dopasowanie innych poza uczelnianych zajęć do naszego planu.
\par Problem ułożenia planu zajęć podejmowany był w wielu pracach naukowych również na Politechnice Gdańskiej, co świadczy o tym że ułożenie bezkonfliktowego planu zajęć jest problemem stosunkowo trudnym do rozwiązania a zarazem bardzo ciekawym, ponieważ w bezpośredni sposób dotyczy każdej uczącej się osoby. Dlatego też na przestrzeni wielu lat powstało wiele różnych podejść do tego problemu od algorytmów klasycznych poprzez różnego typu algorytmy sztucznej inteligencji.
\par Problem układania graficznych rozkładów nie tylko związany jest ze szkolnictwem, ale również dotyczy konieczności ułożenia rozkładów jazdy komunikacji miejskiej \cite{com}
, planu zajęć dla pracowników \cite{worker} oraz terminarza zawodów sportowych \cite{sport}.
\section{Cel pracy}
\par Celem pracy jest zapoznanie się z problemem układania planów zajęć oraz przegląd różnych sposobów rozwiązania tego problemu. Jak również opracowanie, implementacja oraz porównanie działania wybranych przez nas algorytmów do układania planów zajęć : Algorytm roju cząsteczek (Particle Swarm Optimization), Algorytm Genetyczny (Genetic Algorithm) oraz Adaptacyjny Algotym Tabu (Adaptive Tabu Search). Stworzone algorytmy mają układać plan zajęć dla wyspecyfikowanego problemu, uwzględniając ograniczenia miękkie oraz ograniczenia twarde. 
\par Dodatkowo podjęliśmy się implementacji systemu w którym użytkownik będzie miał możliwość wygenerowania planu zajęć wybranym przez siebie algorytmem oraz wyświetlenia stworzonego planu zajęć w Google Calendar.
\par W ramach projektu planowane jest również dostosowanie realnych danych ze szkoly ponadgimnazjalnej do zaimplementowanych przez nas algorytmów.
\section{Zakres pracy}
\section{Podział zadań i obowiązków}
\chapter{Opis problemu układania planu zajęć}
\chapter{Specyfikacja problemu}
\section{Sformułowanie problemu}
\author{Katarzyna Śmietanka} \\
Celem jest stworzenie tygodniowego harmonogramu wykładów dla kilku kursów, z określoną liczbą dostępnych sal i przedziałów czasowych, w których mogą odbywać się zajęcia. Każdy wykład będący w programie danego kursu musi być przypisany do określonego przedziału czasu i sali tak by spełniał wejściowe ograniczenia. 
\subsection{Jednostki problemu}
\begin{itemize}
\item{\textbf{Dni, przedziały, okresy} - dzień podzielony jest na określoną liczbę przedziałów czasowych, okres para złożona z dnia i przedziału czasowego.}
\item{\textbf{Kursy i wykładowcy} - każdy kurs składa się z określonej liczby wykładów, które muszą być rozłożone w różnym czasie, na które uczęszcza określona liczna studentów i prowadzone są one przez wykładowcę. Dla każdego kursu jest określona minimalna liczba dni w których te wykłady muszą się odbyć, oraz okresy w których dane wykłady nie mogą się odbywać.}
\item{\textbf{Sale wykładowe} - sale wykładowe mają ograniczoną liczbę dostępnych miejsc w pomieszczeniu.}
\item{\textbf{Program nauczania} - składa się z kilku kursów, na które uczęszcza grupa studentów.}
\end{itemize}
\subsection{Ograniczenia}
\subsubsection{Ograniczenia twarde}
Plan jest wykonywalny - czyli możliwy do realizacji, jeżeli żadne z wymienionych poniżej ograniczeń nie jest naruszone.
\begin{itemize}
\item  ${H_{1}}$ \textbf{Wykłady} - każdy kurs wchodzący w skład programu nauczania musi być przypisany do różnego okresu.
\item  ${H_{2}}$ \textbf{Zajętość sali} - żadne dwa wykłady nie mogą odbywać się w tym samym okresie w jednym pomieszczeniu.
\item  ${H_{3}}$ \textbf{Konflikty pomiędzy kursami} - zajęcia z tego samego programu nauczania bądź nauczanie przez tego samego wykładowcę muszą odbywać się w różnym czasie.
\item  ${H_{4}}$ \textbf{Dostępność wykładowcy} - zajęcia nie mogą się odbywać w czasie, w którym dany wykładowca nie może prowadzić zajęć.
\end{itemize}
\subsubsection{Ograniczenia miękkie}
Ograniczenia te nie wpływają w bezpośredni sposób na wykonalność planu zajęć, ale na jego jakość uwzględniając poniższe kryteria, które wpływają na funkcję oceny wygenerowanego planu zajęć.
\begin{itemize}
\item  ${S_{1}}$ \textbf{Wielkość sali} - liczba studentów uczęszczajacych na zajęcia w danej sali musi być mniejsza bądź równa liczbie dostępnych miejsc.
\item  ${S_{2}}$ \textbf{Stabilność pomieszczenia} - zajęcia wchodzące w skład jednego kursu powinny odbywać się w jednej tej samej sali, jeżeli jest to niemożliwe liczba sal w których obdywają się te zajęcia powinna być jak najmniejsza.
\item  ${S_{3}}$ \textbf{Minimalna liczba dni} - minimalna liczba dni na które powinny być rozłożone zajęcia z danego kursu.
\item  ${S_{4}}$ \textbf{Zwartość zajęć} - zajęcia wchodzące w skład jednego kierunku powinny być w jak najmniejszych odstępach czasu pomiędzy sobą.
\end{itemize}
\subsection{Funkcja oceny}
\textbf{Ograniczenia twarde} - dla tego typu ograniczeń zliczane są podczas końcowej oceny poszczególne naruszenia ograniczań:\\
\begin{enumerate}
\item \textbf{Zajęcia / Wykłady} - naruszenie występuje w przypadku gdy zajęcia nie są przypisane do planu zajęć
\item \textbf{Zajętość sali} - naruszenie występuje gdy zostaną przypisane więcej niż jedne zajęcia do sali w tym samym czasie
\item \textbf{Konflikty} - naruszenie wystąpi wtedy gdy dwa zajęcia będące w konflikcie odbywają się w tym samym czasie (tzn. ci sami studenci uczęszczają na te zajęcia lub prowadzone są przez tego samego wykładowcę)
\item \textbf{Dostępność} - naruszenie występuje gdy zajęcia odbywają się w czasie, w którym niedostępny jest wykładowca 
\end{enumerate} 

\textbf{Ograniczenia miękkie} - zliczanie punktów kary
\begin{enumerate}
\item \textbf{Wielkość sali} - Jeżeli liczba studentów jest większa niż liczba dostępnych miejsc w sali to za każdego dodatkowego studenta punkt kary pomnożony przez współczynnik ${a_{1} = 1}$ 
\item \textbf{Minimalna liczba dni}
Jeżeli liczba dni podczas których odbywają się zajęcia jest mniejsza niż minimalna liczba dni w których powinny odbywać się zajęcia to do kary doliczamy rożnicę między minimalną liczbą dni a dni w których zajęcia się odbywają pomnożoną o współczynnik $a_{2} = 5$ 
\item \textbf{Zwartość zajęc}
Za każde zajęcia w planie należące do danego programu nauczania nieprzylegające do innych zajęć punkt kary pomnożony o współczynnik ${a_{3} = 2}$
\item \textbf{Stabilność pomieszczenia}
Te same zajęcia powinny odbywać się w jak najmniejszej liczbie różnych pomieszczań, za każde nowe pomieszczenie punkt kary pomnożony o współczynnik ${a_{2} = 1}$
\end{enumerate}

\subsection{Matematyczne sformułowanie problemu} \cite{tabu}
\par Na problem składa się ${n}$ kursów ${C = \{c_{1}, c_{2},...,c_{n}\}}$ które powinny być przydzielone do ${p}$ różnych okresów ${T = \{t_{1}, t_{2},...,t_{p}\}}$ oraz ${m}$ pomieszczeń w których mogą odbywać się zajęcia ${R = \{r_{1}, r_{2},...,r_{m}\}}$. Okres jest to para składająca się z dnia i przedziału czasowego (${d}$ - liczba dni a ${h}$ - liczba dziennych przedziałów czasowych, czyli ${p = d * h}$). Każdy z kursów składa się z ${n}$ zajęć ${L = \{l_{1},l_{2},...,l_{n}\}}$. Kursy wchodzą w skład ${s}$ programów nauczania ${CR = \{cr_{1}, cr_{2}, ..., cr_{s}\}}$, na program nauczania składają się kursy, na które uczęszczają ci sami studenci. 

\chapter{Opis zrealizowanych algorytmów}
\section{Adaptive tabu search}
\author{Tomasz Dziopa, Katarzyna Śmietanka}
\subsection{Specyficzne sformułowanie problemu}
\par Problem definiujemy w postaci macierzy ${X}$  rozmiaru ${p \times m}$ gdzie ${x_{i,j}}$ który definiuje przypisanie danych zajęć do ${t_{j}}$ okresu oraz sali ${r_{i}}$. Jeżeli w danym czasie w danej sali nie odbywają się zajęcia wartość ${x_{i,j}}$ będzie przyjmowała wartość ${null}$. Dzięki takiemu sposobowi zdefiniowania problemu nigdy nie zostanie naruszone ograniczenie twarde ${H_{2}}$ dotyczące zajętości sali.
\subsection{Ogólny opis algorytmu}
\par Opis zaimplementowanego przez nas algorytmu został zaczerpnięty z pracy ,,Adaptive TabuSearch for Course Timetabling'' \cite{tabu}
\par Na całość algorytmu składają się trzy fazy: faza inicjalizacji podczas której tworzony jest początkowy plan zajęć, przy pomocy zachłannej heurystyki, faza intensyfikacji, która jest właściwą fazą algorytmu Adaptive Tabu Search, której celem jest optymalizacja funkcji oceny ograniczeń miękkich. Końcowa faza - faza dywersyfikacji gdzie dokonywana jest redukcja naruszeń miękkich, tak aby nie łamać twardych ograniczeń. Na algorytm ten składa się wiele unikalnych cech między innymi struktury sąsiedztwa - podwójne łańcuchy Kempe, operator zaburzeń oraz dynamiczna integracja operacji przeszukiwania tabu z operatorem zaburzeń.

\subsection{Tabu Search}
\par Algorytm Tabu Search został zaprezentowany w 1986 roku przez Freda W. Glovera \cite{glover}. Jest to metaheurystyka, która opiera się na obserwacji, że proces przeszukiwania przestrzeni rozwiązań w poszukiwaniu najlepszego u ludzi i zwierząt opiera się na pamięci krótko- i długoterminowej. 
\par Pamięć krótkoterminowa realizowana jest w postaci listy ruchów zabronionych. W każdej iteracji przeglądamy strukturę sąsiedztwa w poszukiwaniu najlepszego rozwiązania. Sprawdzamy, czy ruch prowadzący do uzyskania najlepszego sąsiada nie znajduje się na liście ruchów zabronionych; jeżeli tak - rozważamy kolejnego najlepszego sąsiada, w przeciwnym wypadku - aktualizujemy rozwiązanie i dodajemy ruch prowadzący do niego jako ruch zabroniony.

\begin{algorithm}[H]
    \caption{Algorytm Tabu Search}
    \begin{algorithmic}
    \STATE{best = current}
    \WHILE{\emph{nie jest spełniony warunek stopu}}
    \STATE{sąsiedztwo = neighborhood(current)}
    \FOR{$i=0$ \TO len(sąsiedztwo)}
    	\IF{$ruch(current, sąsiedztwo[i]) \not \in lista\_tabu$}
    	\STATE $lista\_tabu = lista\_tabu \cup ruch(current, sąsiedztwo[i])$
    	\STATE current = sąsiedztwo[i]
    	\ENDIF
    \ENDFOR
    \ENDWHILE
    \end{algorithmic}
    \end{algorithm}

\subsection{Fazy algorytmu}
\subsubsection{Faza inicjalizacji}
\par W tej fazie tworzony jest wykonywalny plan zajęć czyli nienaruszający ograniczeń twardych ${H_{1} - H_{4}}$. W każdej iteracji wybierane jest jedno z zajęć z kursu i przypisywane do odpowiedniego okresu i pomieszczenia. Całość przydzielania odbywa się na podstawie dwóch heurystyk pierwsza z nich determinuje wybor kursu, który zostanie przypisany do planu zajęć oraz druga zaś określa parę okres-sala.
\par Dla każdego częściowo wykonywalnego planu zajęć ${P}$ (czyli takiego, do którego zostało przydzielone już część zajęć nie naruszając ograniczeń twardych) próbujemy wybrać jedne z zajęć z kursu, który posiada jeszcze nieprzydzielone zajęcia zgodnie z heurystyką ,,Wybór kursu''. Dzięki tej heurystyce uzyskujemy pierwszeństwo w przydzielaniu kursów mających małą liczbę dostępnych okresów do których może być przypisany oraz kursów z dużą liczbą nieprzypisanych zajęć do planu. Druga z heurystyk ,,Wybór okresu'' dotyczy wyboru okresu, do którego zostaną przypisane dane zajęcia. Celem jest wybór takiego okresu, który ma najmniejsze prawdobodobieństwo bycia wybranym w kolejnych krokach, dla kolejnych nieskończonych kursów.

\par \textbf{Oznaczenia}
\begin{enumerate}
\item $ lo_{i}(P)$ - liczba okresów do których mogą być przydzielone zajęcia z kursu $c_{i}$ dla planu ${P}$
\item $ lp_{i}(P)$ - liczba dostępnych par: okres- sala dla kursu ${c_{i}}$ dla planu ${P}$
\item $ lnz_{i}(P)$ - liczba nieprzydzielonych zajęć dla kursu ${c_{i}}$ dla planu ${P}$
\item $ lnk_{i, j}(P)$ - liczba zajęć z nieskończonych kursów, których nie można przydzielić do okresu ${t_{j}}$ po przydzieleniu jednego z zajęć z kursu ${c_{i}}$ do okresu ${t_{i}}$
\item $kom(i, j, k)$ - całkowita kara związana z ograniczeniami miękkimi po wykonalnym wstawieniu zajęć (tzn. nie naruszając ograniczeń $H1 - H4$ ) z kursu $c_{i}$ do okresu ${t_{j}}$ przydzieleniu sali ${r_{k}}$
\end{enumerate}
\par \textbf{Heurystyki}

\begin{enumerate}
  \item Wybór kursu 
  \begin{enumerate}
    \item Wybieramy kurs z najmniejszą wartością współczynnika:\\
     $ w_a(c_{i}) = \frac{lo_{i}(P)}{\sqrt{lnz_{i}(P)}}$
    \item Jeżeli istnieje więcej niż jeden kurs z tą samą wartością współczynnika ${w_a}$ wybieramy kurs z najmniejszym współczynnikiem \\ $ w_b(c_{i}) = lp_{i}(P) * \sqrt{lnz_{i}(P)} $
    \item Jeżeli istnieje więcej niż jeden kurs z tą samą wartością współczynnika $w_b$ to wybieramy kurs ${c_{i}}$ z maksymalną liczbą kursów współdzielonych kursów z tym kursem (tzn. mającym najwięcej konfliktów ze względu na studentów uczęszczających na ten kurs oraz uczących nauczycieli).
  \end{enumerate}
  \item Wybór okresu \\
  Dla każdej dostępnej pary (okres - sala) wybieramy parę z najmiejszą wartością funkcji $g(j, k) = K_{1} * lnk_{i,j}(P) + K_{2} * kom(i, j, k)$ \\
  $K_{1} = 1.0 $ - współczynnik związany z ograniczeniami twardymi \\
  $K_{2} = 0.5 $ - współczynnik związany z ograniczeniami miękkimi
\end{enumerate}



\subsubsection{Faza intensyfikacji}
\par W fazie intensyfikacji zostają wprowadzone struktury sąsiedztwa prostego oraz pojedyncze i podwójne łańcuchy Kempe, w obrębie tych struktur dochodzi do zamian poszczególnych zajęć tak by nie naruszyć ograniczeń twardych. Na algorytm Tabu Search składa się kombinacja połączenia zamian w obrębie tych dwóch struktur, które przeprowadzane są w cyklu token - ring. Celem tej fazy jest ograniczenie funkcji oceny ograniczeń miękkich, nie łamiąc żadnych ograniczeń twardych. Przestrzeń wykonywanych zamian dla poszczególnych zajęć jest ograniczona tylko do wykonywalnych zamian czyli takich, które po wykonaniu nie naruszają ${H1-H4}$.
\paragraph{Struktury sasiedztwa}
\begin{enumerate}
\item \textbf{Podstawowa struktura sąsiedztwa} \\
Jest to struktura, która zawiera wszystkie możliwe zamiany dla pary dwóch zajęć należących do różnych kursów i nie należących do tego samego okresu w planie zajęć. \\
Zamiana jest przypisaniem zajęć $x_{i_{1},j_{1}}$ w miejsce zajęć ${x_{i_{2}, j_{2}}}$ oraz zajęć ${x_{i_{2}, j_{2}}}$ w miejsce zajęć $x_{i_{1},j_{1}}$ \\
Możliwe przypadki zamian
\begin{enumerate}
\item Zamiana pomiędzy dwoma różnymi zajęciami należącymi do dwóch różnych kursów i okresów
\item Przypisanie zajęcia ${x_{i,j}}$ do wolnej pozycji - do okresu dla którego zajęcie ${x_{i,j}}$ nie wchodzi w konflikt z pozostałymi zajęciami w tym okresie (tzn. nie narusza ${H1-H4}$ )
\end{enumerate}
\textbf{Szczegóły implementacyjne}


\item \textbf{Zaawansowana struktura sąsiedztwa} 
\subparagraph{}
Jednym z klasycznych podejść do problemu układania planu zajęć jest podejście grafowe instancję problemu możemy przedstawić jako graf $G(V, E)$, gdzie wierzchołki grafu reprezentują kursy, a krawędzie reprezentują konflikty między kursami, który należy pokolorować na jak najmniejszą liczbę kolorów. 
\subparagraph{}Łańcuchy Kempe zostały zaproponowane jako narzędzie matematyczne, które miało służyć do udowodnienia twierdzenia o czterech kolorach. Mając dany graf $G$ i jego pokolorowanie na co najmniej dwa kolory, łańcuchy Kempe możemy zdefiniować jako maksymalne spójne podgrafy $G$, w których wszystkie wierzchołki mają nadany kolor $a$ lub $b$.
\subparagraph{}W naszym problemie łańcuchami Kempe będą maksymalne spójne podgrafy $G$, które zostały przypisane do okresu $i$ lub $j$. Dla łańcuchów $K_1$ i $K_2$, które zawierają maksymalne podgrafy kursów przypisanych do odpowiednio $i$ i $j$, gdzie $t_i$ i $t_j$ oznaczają wszystkie kursy przypasowane do $i$ i $j$, tworzymy nowe przypasowania:
\[ t_i' = (t_i \setminus  (K_1 \cup K_2)) \cup (t_j \cap (K_1 \cup K_2)) \]
\[ t_j' = (t_j \setminus  (K_1 \cup K_2)) \cup (t_i \cap (K_1 \cup K_2)) \]

Specjalnym przypadkiem jest, gdy jeden z łańcuchów jest pusty, wtedy:
\[ t_i' = (t_i \setminus K) \cup (t_j \cap K)\]
\[ t_j' = (t_j \setminus K) \cup (t_i \cap K)\]

Tak określony ruch można traktować jako rozszerzoną wersję podstawowej struktury sąsiedztwa, gdzie zamieniamy po kilka przypasowań na raz. Przy zamianie musimy uważać, aby nie wykonywać ruchów, które powodują przypasowanie większej liczby kursów, niż jest dostępnych sal. 

%- room assignment

\end{enumerate}

\subsubsection{Faza dywersyfikacji}
\par Jeżeli rozwiązanie nie może zostać poprawione za pomocą algorytmu tabu search uruchamiana jest trzecia faza - faza dywersyfikacji.Głównym jej elementem jest losowy operator zaburzeń mający na celu zniszczenie osiągniętego lokalnego minimum. Początkowo identyfikowane są zajęcia z wysoką karą wynikającą z funkcji oceny  i losowo wybierane są zajęcia dla których zostaną dokonane zamiany sprecyzowane w poprzedniej fazie.
\par W momencie zakończenia fazy intensyfikacji, poszczególne zajęcia ustawiane są w kolejności malejącej ze względu na wysokość funkcji oceny. Z puli $q$ pierwszych zajęć wybierane jest $n$ zajęć, gdzie zajęcie będące na $k$ miejscu w rankingu wybierane zgodnie z rozkładem prawdopodobieństwa $P(k) = k^{-4.0}$. Następnie dokonywane jest $n$ losowych zamian pomiędzy zajęciami (sprecyzowanych w fazie intensyfikacji), ale tylko takich które zawierają przynajmniej jedno z wybranych z rankingu zajęć. \\
\textbf{Szczegóły implementacyjne}
\section{Particle Swarm Optimization (PSO)}
\subsection{Ogólny opis algorytmu}
\par Opis zaimplementowanego algorytmu został zaczerpnięty z pracy ,,Timetable Scheduling Using Particle Swarm Optimization'' \cite{pso}
\par PSO jest metodą bazującą na zachowaniu społeczeństwa. Polega na przeszukiwaniu przestrzeni rozwiązań przy pomocy populacji (roju) rozwiązań kandydujących (nazywanych cząsteczkami). W każdej iteracji każde rozwiązanie kandydujące (cząsteczka) jest aktualizowane na podstawie dwóch najlepszych wartości. Pierwszą z nich jest poprzednia najlepsza pozycja k-tej cząsteczki w i-tej iteracji ${P}^{i}_{k}$. Drugą zaś jest globalna najlepsza pozycja spośród wszystkich cząsteczek ${G}^{i}$ zanotowana pomiędzy pierwszą a i-tą iteracją. Każda cząsteczka jest równoważna z kandydatem na rozwiązanie problemu. Cząsteczka prousza się zgodnie ze swoją prędkością, która bazuje na doświadczeniu owej cząsteczki oraz doświadczeniu innych cząsteczek. PSO zazwyczaj osiąga rozwiązanie bliskie optymalnemu w mniejszej liczbie iteracji niż algorytmy ewolucyjne czy genetyczne.  \subsection{Działaie algorytmu}
\subsubsection{Orginalna wersja}
\par Orginalne PSO można opisać w trzech krokach:

\begin{description}
  \item[Krok 1] \hfill \\
  The first item
  \item[Krok 2] \hfill \\
  The second item
  \item[Krok 3] \hfill \\
  The third etc \ldots
\end{description}

\subsubsection{PSO w problemie układania planu}
\section{Genetic Algorithm}
\subsection{Ogólny opis algorytmu}
\par Algorytm genetyczny jest heurystyką poszukującą rozwiązania problemu, która naśladuje proces naturalnej selekcji w procesie ewolucji. W informatyce należy do szerszej grupy algorytmów ewolucyjnych i zawiera się w dziedzinie sztucznej inteligencji. Jego działanie opiera się na przeszukiwaniu przestrzeń alternatywnych rozwiązań w problemach optymalizacyjnych w celu wyszukania rozwiązań najlepszych. Cały algorytm operuje na grupie (populacji) potencjalnych rozwiązań, których jakość (stopień, w jakim jest bliskie rozwiązania optymalnego) potrafimy ocenić i które zbliżają się w przypominającym ewolucyjny procesie do rozwiązania optymalnego. Tym, co go wyróżnia jest zastosowanie operacji zaczerpniętych z genetyki, takich jak selekcja, krzyżowanie czy mutacja.
\par Algorytmy genetyczne zajmują bardzo ważne miejsce w dziedzinie projektowania i analizy algorytmów. Doskonale sprawdzają się w sytuacji, gdy problem, z którym przychodzi nam się zmierzyć, jest nie do rozwiązania w sposób klasyczny w sensownym czasie. Pozwalają znaleźć sub-optymalne rozwiązanie problemów, których dziedziny nie są łatwe do wyznaczenia. Są powszechnie stosowane tam, gdzie do uzyskania rozwiązania korzystamy z zagadnień sztucznej inteligencji oraz tam, gdzie uzyskanie rozwiązania jest bardzo złożonym problem, natomiast jego ocena jest łatwa i błyskawiczna. Należy zaznaczyć, że algorytm genetyczny nie gwarantuje znalezienia rozwiązania optymalnego, lecz przybliżone. Dla żadnej ilości iteracji lub liczby osobników nie ma pewności, że algorytm osiągnie optymalne rozwiązanie. W zależności od implementacji istnieje większe lub mniejsze ryzyko, iż algorytm utknie w lokalnym minimum i nie będzie w stanie w pełni wyeksplorować przestrzeń rozwiązań. Z tego powodu w bardzo złożonych problemach niemal pewne jest, iż globalne maksimum nie zostanie osiągnięte. Dlatego też zastosowanie tego algorytmu wciąż się zawęża, gdyż wraz z powstawaniem rozwiązań dedykowanych dla konkretnych problemów, algorytm ten najczęściej okazuje się od nich mniej wydajny. Wciąż jednak pozostaje wiele zagadnień, dla których świat nauki nie znalazł jeszcze specjalistycznego rozwiązania, a w takich przypadkach algorytm genetyczny ciągle pozostaje w gronie heurystyk, które stają się pomocne.
\par Problem definiuje środowisko, w którym istnieje pewna populacja osobników. Każdy z nich posiada zestaw informacji, które tworzą określone struktury.
\begin {itemize}
\item \textbf{Genotyp} - przypisany każdemu osobnikowi ogólny zbiór informacji, które tworzą proponowane rozwiązanie oraz są podstawą do utworzenia fenotypu.
\item \textbf{Fenotyp} - to zbiór cech podlegających ocenie funkcji przystosowania modelującej środowisko, zatem określenia, jak dobre jest dane rozwiązanie.
\item \textbf{Chromosom} - to tutaj zakodowany jest fenotyp i ewentualnie dodatkowe informacje pomocnicze dla procesu tworzenia rozwiązania.
\item \textbf{Gen} - Pojedyncza jednostka informacji, z których zbudowany jest chromosom.
\end{itemize}
\par Schemat działania algorytmu prezentuje się w następujący sposób:
\begin{enumerate}
\item Losowana jest pewna populacja początkowa, każdy osobnik przydzielane ma wygenerowane w sposób możliwie losowy przykładowe rozwiązanie.
\item Populacja poddawana jest ocenie (selekcja). Najlepiej przystosowane osobniki biorą udział w procesie reprodukcji.
\item Wybrane osobniki biorą udział w etapie reprodukcji, który odbywa się poprzez  złączanie genotypów dwójki rodziców (krzyżowanie).
\item Przeprowadzana jest mutacja, czyli wprowadzenie drobnych losowych zmian u niektórych osobników.
\item Rodzi się kolejne pokolenie. Aby utrzymać stałą liczbę osobników w populacji te najlepsze według funkcji oceny przystosowania są powielane, a najsłabsze usuwane. Jeżeli nie znaleziono dostatecznie dobrego rozwiązania, algorytm powraca do kroku drugiego. W przeciwnym wypadku wybieramy najlepszego osobnika z populacji - jego genotyp to uzyskany wynik.
\end{enumerate}
\subsection{Historia i zastosowanie}
Sposób działania algorytmów genetycznych nieprzypadkowo przypomina zjawisko ewolucji biologicznej, ponieważ ich twórca John Henry Holland właśnie z biologii czerpał inspiracje do swoich prac. W 1975 roku wydał książkę \emph{"Adaptation in Natural and Artificial Systems"}, w której jako pierwszy wykazał, jak procesy genetyczne mogą mieć zastosowanie wśród rozwiązywania problemów optymalizacyjnych. Specyfika działania algorytmu czyni go bardzo uniwersalnym. Możliwości jego użycia wybiegają poza czystą algorytmikę i znajduje on zastosowanie w bioinformatyce, inżynierii, ekonomii, chemii, matematyce czy fizyce. Konkretnymi przykładami mogą być np. poszukiwanie najbardziej aerodynamicznego kształtu skrzydła samolotu, opracowanie kształtu anteny najlepiej odbierającej fale radiowe albo, tak jak w tym przypadku, problem układania planu zajęć.
\subsection{Fazy algorytmu w implementowanym rozwiązaniu}
\par Opisywany wcześniej schemat działania algorytmu należy przełożyć na problem układania planu zajęć i zdefiniować schematy danych a następnie operacje na nich wykonywane. Poniższa tabela ilustruje jak obiekty ze świata genetyki odwzorowują opisywany problem.
\begin{center}
\begin{tabular}{| l | p{10cm} |}
\hline
populacja & zbiór wszystkich planów zajęć \\ \hline
osobnik & pojedynczy rozkład zajęć wraz z ograniczeniami \\ \hline
genotyp & rozkład zajęć wszystkich kursów \\ \hline
funkcja przystosowania & funkcja oceny planu względem założeń \\ \hline
fenotyp & realna wartość rozwiązania \\ \hline
chromosom & tablica, której indeksy stanowią dostępne przedziały czasowe, a wartości to listy odbywających się wówczas zajęć w postaci pary (pomieszczenie, kurs) \\ \hline
gen & przyporządkowanie w tablicy o identyfikatorze "czas" pary  (pomieszczenie, kurs) \\ \hline
\end{tabular}
\end{center}
\subsubsection{Utworzenie rozwiązania początkowego}
\par Etap ten jest bardzo złożony i polega na stworzeniu przykładowego rozwiązania dla każdego osobnika populacji. Powinny one się różnić między sobą, lecz nie koniecznie muszą spełniać wszystkie twarde ograniczenia. Ponieważ niespełnianie podstawowych warunków jest sankcjonowane bardzo dużymi karami, w procesie ewolucji rozwiązania te zostaną wyparte lub poprawione.
\par Zatem dla każdego osobnika należy przyporządkować wszystkie zajęcia do jakichkolwiek sal i przedziałów czasowych starając się jednocześnie nie naruszać twardych ograniczeń. Losowana jest wpierw kolejność kursów, których zajęcia będą kolejno przyporządkowywane. Dla każdej lekcji staramy się znaleźć czas i miejsce wedle jednej ze strategii:
\begin{enumerate}
\item Wybierz najmniejsze możliwe pomieszczenie, w którym zmieszczą się wszyscy uczestnicy kursu. Jeśli zajęcia należące do kursu nie odbywają się jeszcze w minimalną zakładaną ilość dni, szukaj wolnego terminu wśród pozostałych dni. Jeśli nie udało się, spróbuj z pomieszczeniem następnym w kolejności pod względem rozmiaru. Powtarzaj ten proces dopóki nie znajdziesz wolnego terminu lub nie sprawdzisz wszystkich pomieszczeń.
\item Podobnie jak w pierwszym przypadku, szukaj wolnych terminów dla pomieszczeń, w których pomieszczą się wszyscy studenci, lecz nie bierz pod uwagę dni, w które odbywają się zajęcia. Tutaj także szukamy wolnego terminu aż sprawdzimy wszystkie pomieszczenia, których pojemność jest niemniejsza niż ilość osób biorących udział w zajęciach.
\item Zastosowanie trzeciej strategii jest rozwiązaniem ostatecznym, ponieważ najczęściej wiąże się z naruszeniem twardych ograniczeń.  Wylosuj przedział czasowy i sprawdź, czy istnieje pokój, wolny pokój, który może pomieścić daną grupę. Nie bierz pod uwagę ograniczeń dostępności prowadzącego. W razie niepowodzenia, poszukaj jakiegokolwiek wolnego pomieszczenia w danym przedziale czasowym. Operacje te powtarzaj losując terminy aż znajdziesz wolny pokój.
\end{enumerate}
\par Ostatnią operacją do wykonania w tym kroku jest ocenienie wszystkich wygenerowanych rozwiązań i zapisaniu najlepszego wyniku. Proces ten sprowadza się do sprawdzenia naruszeń ograniczeń twardych i stopnia spełnialności ograniczeń miękkich poprzez nakładanie odpowiednich kar. W ten sposób najlepszym rozwiązaniem staje się to, dla którego suma kar jest najmniejsza.
\subsubsection{Selekcja}
Etap ten polega na dokonaniu wyboru, które osobnika zostaną poddane krzyżowaniu. Zaimplementowane w celu porównania zostały 3 metody selekcji:
\begin{enumerate}
\item \textbf{Selekcja losowa} - rozwiązanie to jest bardzo proste, a mianowicie polega na losowym wybraniu dwóch różnych od siebie osobników. Taki model powoduje, iż rozwiązania gorsze i słabsze mają taką samą szansę na reprodukcję, zatem nie sprzyja to tworzeniu coraz to lepszych potomków.
\item \textbf{Selekcja ruletkowa} - jej nazwa pochodzi od popularnej gry w ruletkę i sposób wyłaniania osobnika bardzo ją przypomina. Można ją zilustrować jako poruszenie "kołem fortuny", w którym każdemu osobnikowi przypisany jest wycinek, którego wielkość jest odwrotnie proporcjonalna do wartości funkcji przystosowania. Dzieję się tak, ponieważ chcemy aby osobniki o mniejszej wartości funkcji oceny miały większą szansę na wylosowanie.
\item \textbf{Selekcja turniejowa} - wyłanianie osobników do procesu krzyżowania odbywa się poprzez pewną formę konkursu. Losowanych jest kilku uczestników (ich ilość jest jednym z parametrów algorytmu, wartość nominalna to 4), a wybierany jest ten spośród nich, który jest najlepiej przystosowany do środowiska. Zawodnicy w danym turnieju oczywiście muszą być różni, lecz po przeprowadzeniu wielu takich rozgrywek, najlepsi z nich powinni zostać wybierani najczęściej.
\end{enumerate}
\par Etap selekcji zostanie przeprowadzony na połowie wszystkich uczestników, co oznacza, że krzyżowaniu zostanie poddana tylko połowa z nich.
\subsubsection{Krzyżowanie}
Proces ten polega na złączeniu w sposób losowy genotypu dwóch rodziców wybranych w poprzednim kroku w celu stworzenie potomka, który będzie dziedziczył po nich wszystkie cechy. Ilość materiału genetycznego każdego z rodziców powinna być jednakowa. W naszym rozwiązaniu, z danej pary rodziców tworzona jest dwójka dzieci będących początkowo kopiami swoich rodziców. Następnie część genotypu pierwszego dziecka jest zastępowana genotypem drugiego rodzica, a drugiemu dziecku wprowadzane są informacje dziedziczone po pierwszym rodzicu. Schemat krzyżowania w kontekście planu zajęć przestawia się w następujący sposób:
\begin{enumerate}
\item Tworzymy kopie rodziców \emph{child1 = mother} oraz \emph{child2 = father}
\item Losujemy kursy, których zajęcia zawarte u rodzica będą wprowadzane do drugiego dziecka w miejsce zajmowane w genotypie rodzica. Ponieważ nie wszystkie zajęcia będzie dało się przenieść tak, aby nie powodowały konfliktów, kryterium końca tego procesu będzie jakiekolwiek przeniesienie połowy sumarycznej ilości kursów.
\item 
\begin{itemize}
\item Dla każdego przenoszonego zajęcia sprawdź, czy dane zajęcia i odpowiadające mu zajęcie u dziecka są identyczne. W takim wypadku nic nie rób i oznacz zamianę jako udaną.
\item Jeśli pomieszczenie jest już zajęte, oznacz operację jako nieudaną.
\item Jeśli pomieszczenie jest wolne i wprowadzenie zajęć w tym terminie nie narusza żadnych ograniczeń twardych, to oznacz operację jako pozytywną, w przeciwnym razie jako negatywną.
\end{itemize}
\item Jeśli wynik poprzedniego kroku jest negatywny postępują według następujących scenariuszy:
\begin{itemize}
\item Jeśli gen, który chcemy wprowadzić nie jest pusty, to usuń jego dowolnego reprezentanta w genotypie dziecka.
\item Jeśli pomieszczenie jest wolne, spróbuj wprowadzić lekcję w wybranym terminie, pod warunkiem, że ta operacja nie narusza ograniczeń twardych. Jeśli się powiedzie, oznacz operację jako udaną, jeśli nie, podążaj dalej.
\item Spróbuj znaleźć inny termin z wolnym tym samym pokojem oraz dla którego operacja wprowadzenia genu nie powoduje konfliktów. W przypadku niepowodzenia, przejdź do kolejnego kroku.
\item Poszukuj jakichkolwiek wolnych pomieszczeń, w których mogą odbyć się dane zajęcia i szukaj dla niego wolnego terminu. Jeśli taka para się nie znajdzie, operacja zostaje ostatecznie oznaczona jako nieudana i należy spróbować ją powtórzyć dla innego kursu i zajęć.
\end{itemize}
\end{enumerate}
\par Cały opisany powyżej schemat zostaje powtórzony w analogiczny sposób dla drugiego dziecka. Umożliwia on losowe wymieszanie genów rodziców przy jednoczesnym zachowaniu ważności rozwiązania. Wiąże się to jednak z koniecznością długich poszukiwań terminów dla wprowadzanych genów. Z każdej pary rodziców powstaje dwójka potomstwa. Zgodnie z zasadami ewolucji, gdzie przetrwać mogą tylko najlepiej przystosowani, do następnego pokolenia przejdzie lepsza połowa osobników z poprzedniego pokolenia, która zostanie uzupełniona dziećmi powstałymi właśnie dziećmi.
\subsubsection{Mutacja}
Podobnie jak w naturalnym procesie mutacji, tutaj też polega on na wprowadzeniu losowych zmian w powstałym kompletnym potomku. Prawdopodobieństwo jego zajścia jest niewielkie (domyślnie 1\%), więc dotyczyć on będzie jednego bądź kilku genów. W tym przypadku skasowane zostaną wszystkie zajęcia z wybranego kursu i algorytm spróbuje na nowo je przyporządkować. Proces ten jest identyczny jak w przypadku generowania rozwiązania początkowego.
\subsubsection{Elityzm}
\par Jest to zagadnienie związane z etapem mutacji. Ponieważ rezultat tego działania może polepszyć lub pogorszyć wcześniejsze rozwiązanie, nie będziemy mu poddawać jednego (domyślnie) lub kilku najlepszych rozwiązań, aby nie utracić ich wyniku. Elityzmem nazywamy więc ochronę najlepszego rozwiązania przed możliwą regresją podczas mutacji.
\subsection{Istniejące implementacje}
\begin{itemize}
\item{FET - Free Evolutionary Timetable}
Jest to program darmowy, stworzony dla platformy Linux, rozpowszechniany na zasadach licencji GNU GPL. Napisany w C++ z wykorzystaniem biblioteki Qt, dane pobiera z plików w formacie XML. Rozbudowany interfejs umożliwia śledzenie procesu ewolucji.
\item{Tablix}
Jest to polski produkt typu \emph{open source}, jego autorem jest Tomasz Solc. Program powstał w technologii języka C z użyciem pakietu PVM, który umożliwia zrównoleglenie obliczeń, co znacząco przyspiesza działanie algorytmu. Dane wejściowe muszą znajdować się w plikach XML. Nie posiada natomiast interfejsu graficznego, który umożliwiałby edycję danych lub wizualizację wyników.
\end{itemize}
\chapter{Projekt systemu}
\subsection{Zastosowanie}
\subsubsection{Przypadki użycia}
\par
\begin{center}
\includegraphics[width=0.8\textwidth]{InzynierkaUseCase.png}
\end{center}
\subsection{Architektura systemu}
\par
\begin{center}
\includegraphics[width=0.8\textwidth]{ComponentsDiagram.png}
\end{center}
\chapter{Testy}
\chapter{Porównanie algorytmów}
\chapter{Sposób użytkowania systemu}
\chapter{Podsumowanie}

\begin{thebibliography}{10}
\bibitem{tabu} Zhipeng Lu, Jin-Kao Hao. Adaptive TabuSearch for Course Timetabling.  \emph{European Journal of Operational Research}, 200(1):235-244, 2010.
\bibitem{com} U. Brannlund, P. ). Lindberg, A. Nou, J. E. Nilsson. Timetabling using lagrangian relaxation Transportation Science 32, 358-369, 1998
\bibitem{sport} J. A. M. Schreuder. Constructing timetables for sport competitions. \emph{Mathematical Programming Study 13}, 58-67, 1980
\bibitem{worker} M. Chiarandini, A. Schaerf, F. Tiozzo. Solving employee timetabling problems with flexible workload using tabu search. \emph{Proceedings of the 3rd PATAT Conference}, 2013
\bibitem{pso} Autor1, Autor1. Tytuł.  \emph{European Journal of Operational Research}, 200(1):235-244, 2010.

\bibitem{glover} Fred Glover (1986). ,,Future Paths for Integer Programming and Links to Artificial Intelligence'' \emph{Computers and Operations Research}, 13 (5): 533–549.
\bibitem{Mitchell} Mitchell Melanie (1999). \emph{An Introduction to Genetic Algorithms}, A Bradford Book The MIT Press, ISBN 0-262-63185-7
\bibitem{implementacjeGA} Marek Jaszuk. \emph{Zastosowanie algorytmów genetycznych do układania planu zajęć}, dostępne pod adresem, \url{http://www.kmis.pwsz.chelm.pl/publikacje/III/Jaszuk.pdf}
\bibitem{FET} \url{http://www.lalescu.ro/liviu/fet}
\bibitem{Tablix} \url{http://www.tablix.org}
\end{thebibliography}

\end{document}
