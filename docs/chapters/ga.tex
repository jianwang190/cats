\section{Genetic Algorithm (GA)}
\author{Filip Czajkowski}
\subsection{Ogólny opis algorytmu}
\par Algorytm genetyczny jest heurystyką poszukującą rozwiązania problemu, która naśladuje proces naturalnej selekcji w procesie ewolucji. W informatyce należy do szerszej grupy algorytmów ewolucyjnych i zawiera się w dziedzinie sztucznej inteligencji. Jego działanie opiera się na przeszukiwaniu przestrzeń alternatywnych rozwiązań w problemach optymalizacyjnych w celu wyszukania rozwiązań najlepszych. Cały algorytm operuje na grupie (populacji) potencjalnych rozwiązań, których jakość (stopień, w jakim jest bliskie rozwiązania optymalnego) potrafimy ocenić i które zbliżają się w przypominającym ewolucyjny procesie do rozwiązania optymalnego. Tym, co go wyróżnia jest zastosowanie operacji zaczerpniętych z genetyki, takich jak selekcja, krzyżowanie czy mutacja.
\par Algorytmy genetyczne zajmują bardzo ważne miejsce w dziedzinie projektowania i analizy algorytmów. Doskonale sprawdzają się w sytuacji, gdy problem, z którym przychodzi nam się zmierzyć, jest nie do rozwiązania w sposób klasyczny w sensownym czasie. Pozwalają znaleźć sub-optymalne rozwiązanie problemów, których dziedziny nie są łatwe do wyznaczenia. Są powszechnie stosowane tam, gdzie do uzyskania rozwiązania korzystamy z zagadnień sztucznej inteligencji oraz tam, gdzie uzyskanie rozwiązania jest bardzo złożonym problem, natomiast jego ocena jest łatwa i błyskawiczna. Należy zaznaczyć, że algorytm genetyczny nie gwarantuje znalezienia rozwiązania optymalnego, lecz przybliżone. Dla żadnej ilości iteracji lub liczby osobników nie ma pewności, że algorytm osiągnie optymalne rozwiązanie. W zależności od implementacji istnieje większe lub mniejsze ryzyko, iż algorytm utknie w lokalnym minimum i nie będzie w stanie w pełni wyeksplorować przestrzeń rozwiązań. Z tego powodu w bardzo złożonych problemach niemal pewne jest, iż globalne maksimum nie zostanie osiągnięte. Dlatego też zastosowanie tego algorytmu wciąż się zawęża, gdyż wraz z powstawaniem rozwiązań dedykowanych dla konkretnych problemów, algorytm ten najczęściej okazuje się od nich mniej wydajny. Wciąż jednak pozostaje wiele zagadnień, dla których świat nauki nie znalazł jeszcze specjalistycznego rozwiązania, a w takich przypadkach algorytm genetyczny ciągle pozostaje w gronie heurystyk, które stają się pomocne.
\par Problem definiuje środowisko, w którym istnieje pewna populacja osobników. Każdy z nich posiada zestaw informacji, które tworzą określone struktury.
\begin {itemize}
\item \textbf{Genotyp} - przypisany każdemu osobnikowi ogólny zbiór informacji, które tworzą proponowane rozwiązanie oraz są podstawą do utworzenia fenotypu.
\item \textbf{Fenotyp} - to zbiór cech podlegających ocenie funkcji przystosowania modelującej środowisko, zatem określenia, jak dobre jest dane rozwiązanie.
\item \textbf{Chromosom} - to tutaj zakodowany jest fenotyp i ewentualnie dodatkowe informacje pomocnicze dla procesu tworzenia rozwiązania.
\item \textbf{Gen} - Pojedyncza jednostka informacji, z których zbudowany jest chromosom.
\end{itemize}
\par Schemat działania algorytmu prezentuje się w następujący sposób:
\begin{enumerate}
\item Losowana jest pewna populacja początkowa, każdy osobnik przydzielane ma wygenerowane w sposób możliwie losowy przykładowe rozwiązanie.
\item Populacja poddawana jest ocenie (selekcja). Najlepiej przystosowane osobniki biorą udział w procesie reprodukcji.
\item Wybrane osobniki biorą udział w etapie reprodukcji, który odbywa się poprzez  złączanie genotypów dwójki rodziców (krzyżowanie).
\item Przeprowadzana jest mutacja, czyli wprowadzenie drobnych losowych zmian u niektórych osobników.
\item Rodzi się kolejne pokolenie. Aby utrzymać stałą liczbę osobników w populacji te najlepsze według funkcji oceny przystosowania są powielane, a najsłabsze usuwane. Jeżeli nie znaleziono dostatecznie dobrego rozwiązania, algorytm powraca do kroku drugiego. W przeciwnym wypadku wybieramy najlepszego osobnika z populacji - jego genotyp to uzyskany wynik.
\end{enumerate}
\subsection{Historia i zastosowanie}
Sposób działania algorytmów genetycznych nieprzypadkowo przypomina zjawisko ewolucji biologicznej, ponieważ ich twórca John Henry Holland właśnie z biologii czerpał inspiracje do swoich prac. W 1975 roku wydał książkę \emph{"Adaptation in Natural and Artificial Systems"}, w której jako pierwszy wykazał, jak procesy genetyczne mogą mieć zastosowanie wśród rozwiązywania problemów optymalizacyjnych. Specyfika działania algorytmu czyni go bardzo uniwersalnym. Możliwości jego użycia wybiegają poza czystą algorytmikę i znajduje on zastosowanie w bioinformatyce, inżynierii, ekonomii, chemii, matematyce czy fizyce. Konkretnymi przykładami mogą być np. poszukiwanie najbardziej aerodynamicznego kształtu skrzydła samolotu, opracowanie kształtu anteny najlepiej odbierającej fale radiowe albo, tak jak w tym przypadku, problem układania planu zajęć.
\subsection{Fazy algorytmu w implementowanym rozwiązaniu}
\par Opisywany wcześniej schemat działania algorytmu należy przełożyć na problem układania planu zajęć i zdefiniować schematy danych a następnie operacje na nich wykonywane. Poniższa tabela ilustruje jak obiekty ze świata genetyki odwzorowują opisywany problem.
\begin{center}
\begin{tabular}{| l | p{10cm} |}
\hline
populacja & zbiór wszystkich planów zajęć \\ \hline
osobnik & pojedynczy rozkład zajęć wraz z ograniczeniami \\ \hline
genotyp & rozkład zajęć wszystkich kursów \\ \hline
funkcja przystosowania & funkcja oceny planu względem założeń \\ \hline
fenotyp & realna wartość rozwiązania \\ \hline
chromosom & tablica, której indeksy stanowią dostępne przedziały czasowe, a wartości to listy odbywających się wówczas zajęć w postaci pary (pomieszczenie, kurs) \\ \hline
gen & przyporządkowanie w tablicy o identyfikatorze "czas" pary  (pomieszczenie, kurs) \\ \hline
\end{tabular}
\end{center}
\subsubsection{Utworzenie rozwiązania początkowego}
\par Etap ten jest bardzo złożony i polega na stworzeniu przykładowego rozwiązania dla każdego osobnika populacji. Powinny one się różnić między sobą, lecz nie koniecznie muszą spełniać wszystkie twarde ograniczenia. Ponieważ niespełnianie podstawowych warunków jest sankcjonowane bardzo dużymi karami, w procesie ewolucji rozwiązania te zostaną wyparte lub poprawione.
\par Zatem dla każdego osobnika należy przyporządkować wszystkie zajęcia do jakichkolwiek sal i przedziałów czasowych starając się jednocześnie nie naruszać twardych ograniczeń. Losowana jest wpierw kolejność kursów, których zajęcia będą kolejno przyporządkowywane. Dla każdej lekcji staramy się znaleźć czas i miejsce wedle jednej ze strategii:
\begin{enumerate}
\item Wybierz najmniejsze możliwe pomieszczenie, w którym zmieszczą się wszyscy uczestnicy kursu. Jeśli zajęcia należące do kursu nie odbywają się jeszcze w minimalną zakładaną ilość dni, szukaj wolnego terminu wśród pozostałych dni. Jeśli nie udało się, spróbuj z pomieszczeniem następnym w kolejności pod względem rozmiaru. Powtarzaj ten proces dopóki nie znajdziesz wolnego terminu lub nie sprawdzisz wszystkich pomieszczeń.
\item Podobnie jak w pierwszym przypadku, szukaj wolnych terminów dla pomieszczeń, w których pomieszczą się wszyscy studenci, lecz nie bierz pod uwagę dni, w które odbywają się zajęcia. Tutaj także szukamy wolnego terminu aż sprawdzimy wszystkie pomieszczenia, których pojemność jest niemniejsza niż ilość osób biorących udział w zajęciach.
\item Zastosowanie trzeciej strategii jest rozwiązaniem ostatecznym, ponieważ najczęściej wiąże się z naruszeniem twardych ograniczeń.  Wylosuj przedział czasowy i sprawdź, czy istnieje pokój, wolny pokój, który może pomieścić daną grupę. Nie bierz pod uwagę ograniczeń dostępności prowadzącego. W razie niepowodzenia, poszukaj jakiegokolwiek wolnego pomieszczenia w danym przedziale czasowym. Operacje te powtarzaj losując terminy aż znajdziesz wolny pokój.
\end{enumerate}
\par Ostatnią operacją do wykonania w tym kroku jest ocenienie wszystkich wygenerowanych rozwiązań i zapisaniu najlepszego wyniku. Proces ten sprowadza się do sprawdzenia naruszeń ograniczeń twardych i stopnia spełnialności ograniczeń miękkich poprzez nakładanie odpowiednich kar. W ten sposób najlepszym rozwiązaniem staje się to, dla którego suma kar jest najmniejsza.
\subsubsection{Selekcja}
Etap ten polega na dokonaniu wyboru, które osobnika zostaną poddane krzyżowaniu. Zaimplementowane w celu porównania zostały 3 metody selekcji:
\begin{enumerate}
\item \textbf{Selekcja losowa} - rozwiązanie to jest bardzo proste, a mianowicie polega na losowym wybraniu dwóch różnych od siebie osobników. Taki model powoduje, iż rozwiązania gorsze i słabsze mają taką samą szansę na reprodukcję, zatem nie sprzyja to tworzeniu coraz to lepszych potomków.
\item \textbf{Selekcja ruletkowa} - jej nazwa pochodzi od popularnej gry w ruletkę i sposób wyłaniania osobnika bardzo ją przypomina. Można ją zilustrować jako poruszenie "kołem fortuny", w którym każdemu osobnikowi przypisany jest wycinek, którego wielkość jest odwrotnie proporcjonalna do wartości funkcji przystosowania. Dzieję się tak, ponieważ chcemy aby osobniki o mniejszej wartości funkcji oceny miały większą szansę na wylosowanie.
\item \textbf{Selekcja turniejowa} - wyłanianie osobników do procesu krzyżowania odbywa się poprzez pewną formę konkursu. Losowanych jest kilku uczestników (ich ilość jest jednym z parametrów algorytmu, wartość nominalna to 4), a wybierany jest ten spośród nich, który jest najlepiej przystosowany do środowiska. Zawodnicy w danym turnieju oczywiście muszą być różni, lecz po przeprowadzeniu wielu takich rozgrywek, najlepsi z nich powinni zostać wybierani najczęściej.
\end{enumerate}
\par Etap selekcji zostanie przeprowadzony na połowie wszystkich uczestników, co oznacza, że krzyżowaniu zostanie poddana tylko połowa z nich.
\subsubsection{Krzyżowanie}
Proces ten polega na złączeniu w sposób losowy genotypu dwóch rodziców wybranych w poprzednim kroku w celu stworzenie potomka, który będzie dziedziczył po nich wszystkie cechy. Ilość materiału genetycznego każdego z rodziców powinna być jednakowa. W naszym rozwiązaniu, z danej pary rodziców tworzona jest dwójka dzieci będących początkowo kopiami swoich rodziców. Następnie część genotypu pierwszego dziecka jest zastępowana genotypem drugiego rodzica, a drugiemu dziecku wprowadzane są informacje dziedziczone po pierwszym rodzicu. Schemat krzyżowania w kontekście planu zajęć przestawia się w następujący sposób:
\begin{enumerate}
\item Tworzymy kopie rodziców \emph{child1 = mother} oraz \emph{child2 = father}
\item Losujemy kursy, których zajęcia zawarte u rodzica będą wprowadzane do drugiego dziecka w miejsce zajmowane w genotypie rodzica. Ponieważ nie wszystkie zajęcia będzie dało się przenieść tak, aby nie powodowały konfliktów, kryterium końca tego procesu będzie jakiekolwiek przeniesienie połowy sumarycznej ilości kursów.
\item 
\begin{itemize}
\item Dla każdego przenoszonego zajęcia sprawdź, czy dane zajęcia i odpowiadające mu zajęcie u dziecka są identyczne. W takim wypadku nic nie rób i oznacz zamianę jako udaną.
\item Jeśli pomieszczenie jest już zajęte, oznacz operację jako nieudaną.
\item Jeśli pomieszczenie jest wolne i wprowadzenie zajęć w tym terminie nie narusza żadnych ograniczeń twardych, to oznacz operację jako pozytywną, w przeciwnym razie jako negatywną.
\end{itemize}
\item Jeśli wynik poprzedniego kroku jest negatywny postępują według następujących scenariuszy:
\begin{itemize}
\item Jeśli gen, który chcemy wprowadzić nie jest pusty, to usuń jego dowolnego reprezentanta w genotypie dziecka.
\item Jeśli pomieszczenie jest wolne, spróbuj wprowadzić lekcję w wybranym terminie, pod warunkiem, że ta operacja nie narusza ograniczeń twardych. Jeśli się powiedzie, oznacz operację jako udaną, jeśli nie, podążaj dalej.
\item Spróbuj znaleźć inny termin z wolnym tym samym pokojem oraz dla którego operacja wprowadzenia genu nie powoduje konfliktów. W przypadku niepowodzenia, przejdź do kolejnego kroku.
\item Poszukuj jakichkolwiek wolnych pomieszczeń, w których mogą odbyć się dane zajęcia i szukaj dla niego wolnego terminu. Jeśli taka para się nie znajdzie, operacja zostaje ostatecznie oznaczona jako nieudana i należy spróbować ją powtórzyć dla innego kursu i zajęć.
\end{itemize}
\end{enumerate}
\par Cały opisany powyżej schemat zostaje powtórzony w analogiczny sposób dla drugiego dziecka. Umożliwia on losowe wymieszanie genów rodziców przy jednoczesnym zachowaniu ważności rozwiązania. Wiąże się to jednak z koniecznością długich poszukiwań terminów dla wprowadzanych genów. Z każdej pary rodziców powstaje dwójka potomstwa. Zgodnie z zasadami ewolucji, gdzie przetrwać mogą tylko najlepiej przystosowani, do następnego pokolenia przejdzie lepsza połowa osobników z poprzedniego pokolenia, która zostanie uzupełniona dziećmi powstałymi właśnie dziećmi.
\subsubsection{Mutacja}
Podobnie jak w naturalnym procesie mutacji, tutaj też polega on na wprowadzeniu losowych zmian w powstałym kompletnym potomku. Prawdopodobieństwo jego zajścia jest niewielkie (domyślnie 1\%), więc dotyczyć on będzie jednego bądź kilku genów. W tym przypadku skasowane zostaną wszystkie zajęcia z wybranego kursu i algorytm spróbuje na nowo je przyporządkować. Proces ten jest identyczny jak w przypadku generowania rozwiązania początkowego.
\subsubsection{Elityzm}
\par Jest to zagadnienie związane z etapem mutacji. Ponieważ rezultat tego działania może polepszyć lub pogorszyć wcześniejsze rozwiązanie, nie będziemy mu poddawać jednego (domyślnie) lub kilku najlepszych rozwiązań, aby nie utracić ich wyniku. Elityzmem nazywamy więc ochronę najlepszego rozwiązania przed możliwą regresją podczas mutacji.

