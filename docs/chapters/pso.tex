
\section{Particle Swarm Optimization (PSO)}
\author{Paweł Jastrzębski}
\subsection{Ogólny opis algorytmu}
\par Particle Swarm Optimaliation (PSO) lub optymalizacja rojem cząsteczek jest algorytmem zaproponowanym przez Jamesa Kennedy'ego oraz Russela Eberhart'a w roku 1995. Jest to technika wzorowana na zachowaniach występujących w przyrodzie. Algorytm naśladuje sposób poruszania się i inteligencje roju owadów poszukujących pożywienia. 
\par Zachowania te są w pewien sposób upraszczane, zamiast roju owadów mamy pewną liczbę cząsteczek (agentów) poruszających się w n-wymiarowej przestrzeni. Cząsteczki przemieszczają się w różnych kierunkach poszukując optymalnego rozwiązania. Korzystają przy tym ze swoich indywidualnych doświadczeń jak i doświadczenia ogółu.
\subsection{Działanie algorytmu}
\subsubsection{Oryginalna wersja}
$\cdots$
\subsubsection{PSO w problemie układania planu}
\par Modyfikacja PSO przedstawiona w tym rozdziale została zaczerpnięta z pracy ,,Timetable Scheduling Using Particle Swarm Optimization'' \cite{pso}
\par W tym podejściu każda cząsteczka będzie posiadała dwa kompletne plany zajęć. Jeden będzie planem przeznaczonym do modyfikacji w każdej iteracji algorytmu (Dalej nazywany planem aktualnym). Natomiast drugi będzie używany do pamiętania dotychczas najlepszego planu znalezionego przez tą cząsteczke (Dalej nazywany lokalnie najlepszym planem). Obecny będzie także globalny plan którego zadaniem będzie przechowywanie najpelszego planu znalezionego kiedykolwiek przez jakąkolwiek cząsteczke (Dalej nazywany globalnie najlepszym planem).  
\par Podczas każdej iteracji poszczególne cząsteczki będą poddawane trzem zmianom. Najpierw dwie losowe lekcje z planu aktualnego zostaną ze sobą zamienione. Potem jedna lekcja z lokalnie najlepszego planu zostanie skopjowana do aktualnego palanu. Na koniec skopjujemy jedną lekcje z globalnie najlepszego planu do aktualnego planu. Kopja lekcji z innego planu wykonywana jest w następujący sposób. Wybierana jest lekcja z innego planu. Odszukujemy miejsce gdzie jest ta lekcja w aktualnym planie. Zamieniamy odszukaną lekcjie z lekcją która jest w miejscu do którego chcemy ją skopjować. 
\par Cały algorytm zaczynamy od stworzenia populacji dwudziestu cząsteczek. Każda z nich będzie posiadała losowo wygenerowany plan zajęć. Następnie dla każdej iteracji na każdej cząsteczce wykonwane będą poniższe kroki:
\begin{description}
  \item[Krok 1] \hfill \\
     \par Ocena rozwiązania. \hfill \\
   \par Oceniany zostaje aktualny plan. W tym momęcie następuje aktualizacja lokalnie najlepszego planu oraz globalnie najlepszego planu.
  \item[Krok 2] \hfill \\
     \par Lokalna zamiana lekcji. \hfill \\
    \par Wykonana zostaje losowa zamiana dwóch lekcji w aktualnym planie.

  \item[Krok 3] \hfill \\
      \par Kopia lekcji z lokalnie najlepszego planu. \hfill \\
        \par Losowo wybrana lekcji z lokalnie najlepszego planu zostaje kopjowana do aktualnego planu 
  \item[Krok 4] \hfill \\
      \par Kopia lekcji z globalnie najlepszego planu. \hfill \\
        \par Losowo wybrana lekcji z globalnie najlepszego planu zostaje kopjowana do aktualnego planu 
\end{description}
\par Iteracje powtarzamy do czasu aż nie osiągniemy wystarczająco dobrego planu lub osiągniemy maksymalną liczbę iteracji albo skończy nam się limit czasu.
\subsection{Ewojucja ostatecnego rozwiązania}
