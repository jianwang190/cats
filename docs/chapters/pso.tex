
\section{Particle Swarm Optimization (PSO)}
\author{Paweł Jastrzębski}
\subsection{Ogólny opis algorytmu}
\par Particle Swarm Optimaliation (PSO) lub optymalizacja rojem cząsteczek jest algorytmem zaproponowanym przez Jamesa Kennedy'ego oraz Russela Eberhart'a w roku 1995. Jest to technika wzorowana na zachowaniu występującym w przyrodzie. Algorytm naśladuje sposób poruszania się i inteligencje roju owadów poszukujących pożywienia. 
\par Zachowania te są w pewien sposób upraszczane, zamiast roju owadów mamy pewną liczbę cząsteczek (agentów) poruszających się w n-wymiarowej przestrzeni. Cząsteczki przemieszczają się w różnych kierunkach poszukując optymalnego rozwiązania. Korzystają przy tym ze swoich indywidualnych doświadczeń jak i doświadczenia ogółu.

\subsection{Działanie algorytmu}
\subsubsection{Oryginalna wersja}
\subsubsection{PSO w problemie układania planu}
