\paragraph{}
\par Realizacja projektu inżynierskiego postawiła nas przed ciekawą możliwością głębszego zapoznania się z problemem układania planów zajęć, zagłębienia się w literaturę tematu oraz implementację wybranych przez nas algorytmów: Algorytmu Genetycznego, Algorytmu Adaptacyjnego Tabu oraz Algorytmu Roju Cząsteczek. Zagadnienie naszej pracy inżynierskiej poszerzyliśmy o te dwa dodatkowe algorytmy, dzięki którym mogliśmy porównać działanie Algorytmu Genetycznego w stosunku do innych podejść również nakierowanych na znalezienie opytmalnego, wykonywalnego planu zajęć
\par Pierwotną, planowaną wizją było przystosowanie zaimplementowanych algorytmów do systemu Politechniki Gdańskiej moja.pg. Niestety mimo licznych trudności związanych z prawem dotyczącym ochrony danych osobowych nie udało się uzyskać niezbędnych danych do realizacji projektu. W konsekwencji zdecydowaliśmy się na przystosowanie naszych algorytmów do danych uzyskanych z jednej ze szkół ponadgimnazjalnej, w celu urealnienia tego problemu. Większość wstępnych założeń dotycząca implementacji algorytmów została zrealizowana, a uzyskane przez nas wyniki potwierdziły nasze przypuszczenia co do skuteczności poszczególnych algorytmów. 
\paragraph{}Głównym celem projektu inżynierskiego było stworzenie systemu do automatyzacji procesu układania planu zajęć. W naszej pracy większy nacisk położyliśmy na porównanie trzech algorytmów, które miały w zamyśle służyć jako podstawa systemu. Stworzony przez nas system posiada założone przez nas wymagania, jednak patrząc pod kątem biznesowym, jego funkcjonalność jest dosyć ograniczona. Struktura projektu umożliwia rozbudowanie go w przyszłości o dodatkowe funkcjonalności przydatne potencjalnym użytkownikom końcowym (np. dyrektorom szkół, administratorom uczelni), takimi jak: możliwość edycji wygenerowanego planu, rozproszone zbieranie wymagań od nauczycieli, eksport do większej liczby formatów. Poprzez wydzielenie modułu algorytmów istnieje również możliwość rozbudowania systemu o większą liczbę algorytmów, jak i modyfikacji parametrów działania. 

