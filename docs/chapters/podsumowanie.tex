\paragraph{}
\par Realizacja projektu inżynierskiego postawiła nas przed ciekawą możliwością głębszego zapoznania się z problemem układania planów zajęć, zgłębienia literatury tematu oraz implementację wybranych przez nas algorytmów: genetycznego, adaptacyjnego tabu oraz algorytmu roju cząsteczek. Zagadnienie naszej pracy inżynierskiej poszerzyliśmy o dwa dodatkowe algorytmy, dzięki którym mogliśmy porównać działanie algorytmu genetycznego w stosunku do innych podejść również nakierowanych na znalezienie możliwie najlepszego, wykonywalnego planu zajęć.
\par Pierwotnie planowaną wizją było przystosowanie zaimplementowanych algorytmów do systemu Politechniki Gdańskiej \emph{moja.pg}. Niestety z powodu licznych trudności związanych z prawem dotyczącym ochrony danych osobowych nie udało się uzyskać niezbędnych danych do realizacji projektu. W konsekwencji zdecydowaliśmy się na przystosowanie naszych algorytmów do danych uzyskanych z jednej ze szkół ponadgimnazjalnych, aby sprawdzić nasze rozwiązanie w realnym zastosowaniu. Większość wstępnych założeń dotycząca implementacji algorytmów została zrealizowana, a uzyskane przez nas wyniki potwierdziły nasze przypuszczenia co do skuteczności poszczególnych algorytmów.
\par Wybrana architektura systemu okazała się trafnym rozwiązaniem, gdyż umożliwiła odseparowanie części interfejsu od modułów generujących rozkład zajęć. Pomogło to podzielić pracę pomiędzy członków zespołu a także wytwarzać każdą z części niezależnie od siebie. Także \emph{Python}, jako główna technologia systemu spełniła nasze oczekiwania, gdyż pisany kod był zwięzły i przejrzysty, a jego zastosowanie bardzo szerokie. Jedynym jego mankamentem jest fakt, iż w porównaniu z językiem \emph{C++} czy \emph{Java}, jest dużo wolniejszy, co znacząco wydłuża czas wykonywania kosztownych obliczeniowo działań.
\paragraph{}Głównym celem projektu inżynierskiego było stworzenie systemu do automatyzacji procesu układania planu zajęć. W naszej pracy większy nacisk położyliśmy na porównanie trzech algorytmów, które miały w zamyśle służyć jako podstawa systemu. Stworzony przez nas system posiada założone przez nas wymagania, jednak oceniając jego przydatność biznesową, można stwierdzić, iż jego funkcjonalność jest dosyć ograniczona. Struktura projektu umożliwia rozbudowanie go w przyszłości o dodatkowe funkcjonalności przydatne potencjalnym użytkownikom końcowym (np. dyrektorom szkół, administratorom uczelni), takimi jak: możliwość edycji wygenerowanego planu, rozproszone zbieranie wymagań od nauczycieli, eksport wyników do większej liczby formatów. Poprzez wydzielenie modułu algorytmów istnieje również możliwość rozbudowania systemu o większą liczbę algorytmów, jak i modyfikacji parametrów działania. 

