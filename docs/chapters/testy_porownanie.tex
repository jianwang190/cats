\chapter{Testy i porównanie algorytmów}
\section{Analiza działania poszczególnych algorytmów}
\subsection{Algorytm Roju Cząsteczek}
\subsection{Algorytm Adaptacyjny Tabu}
\subsection{Algorytm Genetyczny}
\section{Specyfikacja testów}
Dane testowe pobrane zostały z konkursu ,,International Timetabling Competition 2007''
\subsection{Test 1}
\begin{table}[H]
\begin{center}

\begin{tabular}{ |c|c|c|c| }
\multicolumn{1}{r}{}
 &  \multicolumn{1}{c}{$$}
 & \multicolumn{1}{c}{$$} 
 \\
\cline{1-2}
$Liczba\ kursów$ & $30$\\
\cline{1-2}
$Liczba\ programów\ nauczania$ & $14$\\
\cline{1-2}
$Liczba\ dni$ & $5$ \\
\cline{1-2}
$Licza\ przedziałów\ czasowych$ & $6$ \\
\cline{1-2}
$Liczba\ ograniczeń$ & $53$ \\
\cline{1-2}
$Liczba\ sal$ & $6$ \\
\cline{1-2}
\end{tabular}
\end{center}
\caption {Specyfikacja danych - Test 1}
\end{table}
\subsubsection{Test 1 - wizualizacja}


\begin{figure}[H]
  \caption{Graf przedstawiający wszystkie zależności pomiędzy kursami (uwzględniając tych samych prowadzących nauczycieli oraz należenie kursu do tego samego programu nauczania) }
  \centering
    \includegraphics[width=0.8\textwidth]{test1.PNG}
\end{figure}


\begin{figure}[H]
  \caption{Graf uwzględniający zależności pomiędzy kursami uwzględniając prowadzących dane kursy}
  \centering
    \includegraphics[width=0.8\textwidth]{test1_teach.PNG}
\end{figure}
\begin{figure}[H]
  \caption{Graf uwględniający grupy zależnych od siebie kursów wchodzących w skład tego samego programu nauczania}
  \centering
    \includegraphics[width=0.8\textwidth]{test1_con.PNG}
\end{figure}
\subsection{Test 2}
\begin{table}[H]
\begin{center}

\begin{tabular}{ |c|c|c|c| }
\multicolumn{1}{r}{}
 &  \multicolumn{1}{c}{$$}
 & \multicolumn{1}{c}{$$} 
 \\
\cline{1-2}
$Liczba\ kursów$ & $82$\\
\cline{1-2}
$Liczba\ programów\ nauczania$ & $70$\\
\cline{1-2}
$Liczba\ dni$ & $5$ \\
\cline{1-2}
$Licza\ przedziałów\ czasowych$ & $5$ \\
\cline{1-2}
$Liczba\ ograniczeń$ & $513$ \\
\cline{1-2}
$Liczba\ sal$ & $16$ \\
\cline{1-2}
\end{tabular}
\end{center}
\caption {Specyfikacja danych - Test 2}
\end{table}
\subsection{Test 3}
\begin{table}[H]
\begin{center}

\begin{tabular}{ |c|c|c|c| }
\multicolumn{1}{r}{}
 &  \multicolumn{1}{c}{$$}
 & \multicolumn{1}{c}{$$} 
 \\
\cline{1-2}
$Liczba\ kursów$ & $72$\\
\cline{1-2}
$Liczba\ programów\ nauczania$ & $68$\\
\cline{1-2}
$Liczba\ dni$ & $5$ \\
\cline{1-2}
$Licza\ przedziałów\ czasowych$ & $5$ \\
\cline{1-2}
$Liczba\ ograniczeń$ & $382$ \\
\cline{1-2}
$Liczba\ sal$ & $16$ \\
\cline{1-2}
\end{tabular}
\end{center}
\caption {Specyfikacja danych - Test 3}
\end{table}






