
\chapter{Wstęp}

\section{Kontekst zagadnienia}
\author{Katarzyna Śmietanka}
\par Problem układania planu zajęć dotyczy wielu instytucji głównie związanych ze szkolnictwem, od szkoły podstawowej aż to szkół wyższych. Zadanie sprowadza się do przygotowania graficznego rozkładu zajęć: dla poszczególnych klas - uczniów uczęszczających do danej klasy, planu zajęć dla każdego z uczących nauczycieli oraz rozkładu zajęć odbywających się w poszczególnych salach. Układanie planów zajęć nawet dla niewielkiej szkoły wymaga dużego wysiłku oraz nakładu czasowego. Problem staje jeszcze bardziej skomplikowany dla szkół ponadgimnazjalnych, do których uczęszcza więcej uczniów a tym samym jest więcej klas, nauczycieli oraz realizowanych przedmiotów. Ponadto zagadnienie ułożenia planu zajęć dla szkół wyższych, w których studenci mogą sami wybierać przedmioty w znacznym stopniu komplikuje te zagadnienie. Trudność problemu związana jest z podstawowymi kryteriami, które musi spełniać plan zajęć aby był właściwy: dwa różne zajęcia, na które zapisany jest minimum jeden student nie mogą się odbywać w tym samym czasie, dwa różne zajęcia w tym samym czasie nie mogą odbywać się w tej samej sali oraz nauczyciel w danym momencie może nauczać tylko jednego przedmiotu. Rozszerzeń tych ograniczeń jest wiele, a uwzględnienie każdego z nich, często konieczne do stworzenia realnego planu zajęć jest niezbędne. Wprowadzanie kolejnych ograniczeń znacznie utrudnia problem, który i tak jest stosunkowo skomplikowany.
\par Pomysł tematu pracy inżynierskiej wyniknął z przyglądania się na przestrzeni naszej edukacji planom zajęć, które często odbiegały od idealnego według naszej opinii. Plan zajęć często zawierał sporo przerw między obowiązkowymi zajęciami, zajęcia zaczynały się bardzo wcześnie lub kończyły bardzo późno, co w znacznym stopniu utrudniało powrót do domu bądź też dopasowanie innych pozauczelnianych zajęć do naszego planu.
\par Problem ułożenia planu zajęć podejmowany był w wielu pracach naukowych również na Politechnice Gdańskiej, co świadczy o tym że ułożenie bezkonfliktowego planu zajęć jest problemem stosunkowo trudnym do rozwiązania a zarazem bardzo ciekawym, ponieważ w bezpośredni sposób dotyczy każdej uczącej się osoby. Dlatego też na przestrzeni wielu lat powstało wiele różnych podejść do tego problemu od algorytmów klasycznych poprzez różnego typu algorytmy sztucznej inteligencji.
\par Problem układania graficznych rozkładów nie tylko związany jest ze szkolnictwem, ale również dotyczy układania rozkładów jazdy komunikacji miejskiej \cite{com}
, planu zajęć dla pracowników \cite{worker} oraz terminarza zawodów sportowych \cite{sport}.
\section{Cel}
\par Celem naszej pracy jest stworzenie systemu do automatyzacji procesu układania planów zajęć z wykorzystaniem Algorytmu Genetycznego (Genetic Algorithm).  W przeciągu ostatnich lat pojawiły się nowe podejścia - algotymy rozwiązujące problem układania planu zajęć, które postawiły algorytm genetyczny w nowym świetle. Chęć głębszego zapoznania się z postawionym przed nami problemem sprawiła, że zdecydowaliśmy się rozszerzenia zagadnienia tematu naszej pracy inżynierskiej o kolejne implementacje algorytmów: Algorytm Roju Cząsteczek (Particle Swarm Optimization) oraz Algorytm Adaptacyjny Tabu (Adaptive Tabu Search). Ponadto w celu urzeczywistnienia problemu zdecydowaliśmy się na przystosowanie działania naszych algorytmów do realych danych uzyskanych z jednej z gdyńskich szkół ponadgimnazjalnych.
\par Implementacja algorytmów: Genetycznego, Roju Cząsteczek oraz Adaptacyjnego Tabu postawiła nas przed ciekawą możliwością porównania działania tych algorytmów, co jak się okazuje nie jest łatwym zadaniem.
\section{Zakres pracy}
\par Główne zakresy pracy:
\begin{enumerate}
\item System do automatyzacji procesu układania planu zajęć
\begin{enumerate}
\item Frontend — aplikacja kliencka, odpowiadająca za interakcję z użytkownikiem: wprowadzanie danych wejściowych, wybór algorytmu do generowania planu zajęć oraz wizualizacja wygenerowanego planu zajęc w API Google Calendar.
\item Backend - generowanie planu zajęć z danych prowadzonych przez użytkownika 
\end{enumerate}
\item Implementacja algorytmów
\begin{enumerate}
\item Algorytm Genetyczny
\item Algorytm Roju Cząsteczek
\item Algorytm Adaptacyjny Tabu
\end{enumerate}
\item Przystosowanie danych szkolnych do zaimplementowanych algorytmów
\item Porównanie działania algorytmów



\end{enumerate}


\section{Podział zadań i obowiązków}
\section{Środowisko pracy i narzędzia}
