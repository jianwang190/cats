\chapter{Specyfikacja problemu}
\section{International Timetabling Competition 2007}
\textit{Filip Czajkowski} \\
\subsection{Opis konkursu}
\par International Timetabling Competition (ITC) to międzynarodowy konkurs w dziedzinie układania planów zajęć organizowany przez międzynarodowe środowisko akademickie. Jego celem jest lepsze poznanie problemu rozkładania zajęć lub egzaminów na uczelni, z którym zmagają się od bardzo dawna i wciąż potrzebują stosowania lepszych rozwiązań. W 2007 roku odbyła się jego druga edycja, w której można było startować w dowolnej z trzech kategorii: 
\begin{itemize}
\item Układanie rozkładu sesji egzaminacyjnej na uczelni. W tym przypadku zakłada się, że wszyscy studenci zapisani na swoje zajęcia muszą mieć możliwość wzięcia udziału we wszystkich dotyczących ich egzaminach. Plan powstaje zatem po zapisaniu się wszystkich studentów na wybrane przez siebie przedmioty.
\item Tworzenie planu zajęć dla studentów zapisanych na konkretne zajęcia. W tym przypadku rozkład wszystkich zajęć obowiązujących przez cały semestr jest generowany w momencie, gdy studenci wybrali już przedmioty, na które zajęcia chcą uczęszczać. Ten model rozwiązania jest dostosowany do elastycznych programów nauczania, które uczelnie coraz częściej stosują.
\item Układanie planu zajęć dla każdego planu nauczania. Tworzony jest tygodniowy rozkład lekcji, które są przyporządkowane do poszczególnych programów nauczania, a także muszą spełniać ograniczenia czasowe narzucone przez władze uczelni. Udostępnione przykładowe dane przedmiotów, sal oraz wykładowców pochodzą z Uniwersytetu Udine we Włoszech.
\end{itemize}
W przewidzianym przedziale czasowym można było nadsyłać rozwiązania, które można było testować na części udostępnionych zestawach danych. Niedozwolone było stosowanie zrównoleglenia obliczeń w stosowanych rozwiązaniach. Ostatecznie, zwycięzcy w każdej z kategorii zostali ogłoszeni podczas konferencji w Montrealu w 2008 roku i każdy z nich otrzymał nagrodę pieniężną w wysokości 500\pounds .
\subsection{Konkurs a praca inżynierska}
\par Poszukując informacji o implementowanych przez nas algorytmach optymalizacyjnych w kontekście układania planu zajęć, natknęliśmy się na informacje o konkursie ITC w 2007 roku. Wspólnie stwierdziliśmy, że skorzystanie ze zdefiniowanych w nim ograniczeń i wymagań względem generowanego planu a także określona funkcja oceny rozwiązania bardzo pomoże nam zaprojektować nasze algorytmy oraz jednoznacznie je porównać. Nasze implementacje były tworzone pod kątem trzeciej kategorii konkursu, czyli rozkładu zajęć przedmiotów zawierających się we wcześniej zdefiniowanych programach nauczania.
\section{Sformułowanie problemu}
\textit{Katarzyna Śmietanka} \\
Celem jest stworzenie tygodniowego harmonogramu wykładów dla kilku kursów, z określoną liczbą dostępnych sal i przedziałów czasowych, w których mogą odbywać się zajęcia. Każdy wykład będący w programie danego kursu musi być przypisany do określonego przedziału czasu i sali tak by spełniał wejściowe ograniczenia. 
\subsection{Jednostki problemu}
\begin{itemize}
\item{\textbf{Dni, przedziały, okresy} - dzień podzielony jest na określoną liczbę przedziałów czasowych, okres para złożona z dnia i przedziału czasowego.}
\item{\textbf{Kursy i wykładowcy} - każdy kurs składa się z określonej liczby zajęć, które muszą być rozłożone w różnym czasie, na które uczęszcza określona liczna studentów i prowadzone są one przez konkretnego wykładowcę. Dla każdego kursu określona jest minimalna liczba dni, na które powinny być rozłożone zajęcia oraz okresy w których dane wykłady (zajęcia) nie mogą się odbywać.}
\item{\textbf{Sale wykładowe} - sale wykładowe mają ograniczoną liczbę dostępnych miejsc w pomieszczeniu.}
\item{\textbf{Program nauczania} - składa się z kilku kursów, na które uczęszcza grupa studentów.}
\end{itemize}
\subsection{Ograniczenia}
\subsubsection{Ograniczenia twarde}
Plan jest wykonywalny - czyli możliwy do realizacji, jeżeli żadne z wymienionych poniżej ograniczeń nie jest naruszone.
\begin{itemize}
\item  ${H_{1}}$ \textbf{Wykłady} - każdy kurs wchodzący w skład programu nauczania musi być przypisany do różnego okresu.
\item  ${H_{2}}$ \textbf{Zajętość sali} - żadne dwa wykłady nie mogą odbywać się w tym samym okresie w jednym pomieszczeniu.
\item  ${H_{3}}$ \textbf{Konflikty pomiędzy kursami} - zajęcia z tego samego programu nauczania bądź nauczanie przez tego samego wykładowcę muszą odbywać się w różnym czasie.
\item  ${H_{4}}$ \textbf{Dostępność wykładowcy} - zajęcia nie mogą się odbywać w czasie, w którym dany wykładowca nie może prowadzić zajęć.
\item ${H_{5}}$ \textbf{Typ sali} - każde z zajęć powinno odbywać się w odpowiedniej sali przystosowanej do przeprowadzania zajęć (odpowiednim typie sali) [*] - ograniczenie to dotyczy tylko $Szkolne\ dane\ wejściowe$
\end{itemize}
\subsubsection{Ograniczenia miękkie}
Ograniczenia te w bezpośredni sposób nie wpływają na wykonywalność planu zajęć, ale na jakość wygenerowanego planu zajęć, określoną na podstawie funkcji oceny.
\begin{itemize}
\item  ${S_{1}}$ \textbf{Wielkość sali} - liczba studentów uczęszczajacych na zajęcia w danej sali musi być mniejsza bądź równa liczbie dostępnych miejsc.
\item  ${S_{2}}$ \textbf{Stabilność pomieszczenia} - zajęcia wchodzące w skład jednego kursu powinny odbywać się w jednej tej samej sali, jeżeli jest to niemożliwe liczba sal w których obdywają się te zajęcia powinna być jak najmniejsza.
\item  ${S_{3}}$ \textbf{Minimalna liczba dni} - minimalna liczba dni na które powinny być rozłożone zajęcia z danego kursu.
\item  ${S_{4}}$ \textbf{Zwartość zajęć} - zajęcia wchodzące w skład jednego programu nauczania powinny odbywać się w jak najmniejszych odstępach czasowych między sobą. [*] - dotyczy to tylko $Danych\ konkursowych$
\end{itemize}
\subsection{Funkcja oceny - dla danych konkursowych}
\textbf{Ograniczenia twarde} - podczas oceny końcowej wygenerowanego planu zajęć zliczane są poszczególne naruszenia ograniczeń twardych:\\
\begin{enumerate}
\item \textbf{Zajęcia / Wykłady} - naruszenie występuje w przypadku gdy zajęcia nie są przypisane do planu zajęć
\item \textbf{Zajętość sali} - naruszenie występuje gdy zostaną przypisane więcej niż jedne zajęcia do sali w tym samym czasie
\item \textbf{Konflikty} - naruszenie wystąpi wtedy, gdy dwa zajęcia będące w konflikcie odbywają się w tym samym czasie (tzn. ci sami studenci uczęszczają na te zajęcia lub prowadzone są przez tego samego wykładowcę)
\item \textbf{Dostępność} - naruszenie występuje gdy zajęcia odbywają się w czasie, w którym niedostępny jest wykładowca 
\end{enumerate} 

\textbf{Ograniczenia miękkie} - zliczanie punktów kary
\begin{enumerate}
\item \textbf{Wielkość sali} - Jeżeli liczba studentów jest większa niż liczba dostępnych miejsc w sali to za każdego dodatkowego studenta punkt kary pomnożony przez współczynnik ${a_{1} = 1}$ 
\item \textbf{Minimalna liczba dni}
Jeżeli liczba dni podczas których odbywają się zajęcia jest mniejsza niż minimalna liczba dni, podczas których powinny odbywać się zajęcia to do kary doliczamy rożnicę między minimalną liczbą dni a liczbą dni, w których zajęcia się odbywają pomnożoną o współczynnik $a_{2} = 5$ 
\item \textbf{Zwartość zajęc}
Za każde zajęcia w planie, które nie przylegają do żadnych innych zajęć z tego samego programu nauczania, punkt kary pomnożony o współczynnik ${a_{3} = 2}$
\item \textbf{Stabilność pomieszczenia}
Te same zajęcia powinny odbywać się w jak najmniejszej liczbie różnych pomieszczań, za każde nowe pomieszczenie punkt kary pomnożony o współczynnik ${a_{4} = 1}$
\end{enumerate}

\subsection{Matematyczny zapis przestrzeni rozwiązań} \cite{tabu}
\par Na problem składa się ${n}$ kursów ${C = \{c_{1}, c_{2},...,c_{n}\}}$ które powinny być przydzielone do ${p}$ różnych okresów ${T = \{t_{1}, t_{2},...,t_{p}\}}$ oraz ${m}$ pomieszczeń w których mogą odbywać się zajęcia ${R = \{r_{1}, r_{2},...,r_{m}\}}$. Okres jest to para składająca się z dnia i przedziału czasowego (${d}$ - liczba dni a ${h}$ - liczba dziennych przedziałów czasowych, czyli ${p = d * h}$). Każdy z kursów składa się z ${n}$ zajęć ${L = \{l_{1},l_{2},...,l_{n}\}}$. Kursy wchodzą w skład ${s}$ programów nauczania ${CR = \{cr_{1}, cr_{2}, ..., cr_{s}\}}$, na program nauczania składają się kursy, na które uczęszczają ci sami studenci. 

\section{Dane wejściowe}
\label{sec:input_format}
\subsection{Konkursowe dane wejściowe}
\par Dane wejściowe zostały uzyskane z konkursu International Timetabling Competition 2007
\par Dane wejściowe są plikami tekstowymi, w których zawarte są ogólne informacje takie jak: liczba kursów, liczba dostępnych pomieszczeń, liczba dni na które mają być rozłożone zajęcia, liczba programów nauczania oraz ograniczenia precyzujące w jakich terminach nie mogą odbywać się zajęcia z danego kursu. Kolejno sprecyzowane są kursy, pomieszczenia, programy nauczania oraz ograniczenia w zadanym poniżej formacie:
\begin{verbatim}
Kursy: <Id Kurs> <Nauczyciel> <Liczba zajęć> <Minimalna liczba dni> <liczba studentów>
Sale: <Id Sala> <Pojemność sali>
Program nauczania: <Id Program Nauczania> <Liczba Kursów> <Ids Kursy>
Ograniczenia: <Id Kurs> <Dzień> <Okres>
\end{verbatim}
\begin{enumerate}
\item Każdy z kursów z danych wejściowych zawiera unikalny identyfikator, liczbę zajęć, która musi się odbyć w ramach tego kursu, minimalną liczbę dni oraz liczbę studentów, którzy uczęszczają na te zajęcia. 
\item Sale zawierają unikalny identyfikator oraz informację o maksymalnej liczbie studentów, która się zmieści w sali.
\item Na program nauczania składa się określona liczba kursów, których identyfikatory zawarte są w opisie programu nauczania  \verb#<Ids Kursy>#

\item W ramach ograniczeń umieszczona jest informacja o kursie \verb#<Id Kurs>#
 który nie może odbyć się w danym dniu w danym okresie.
\end{enumerate}
\subsection{Szkolne dane wejściowe}
\par Dane z roku szkolnego 2013/2014 uzyskane z jednej ze szkół gdyńskich ponadgimnazjalnych, która jest połączeniem szkoły zawodowej, technikum oraz liceum.
\par Dane zostały przystosowane do formatu ,,Konkursowych danych wejściowych'' rozszerzając je o pojęcie typu sali. Szczegółowy proces przetworzenia danych został opisany w sekcji ,,Etapy przetwarzania danych szkolnych''.
\begin{verbatim}
Kursy: <Id Kurs> <Nauczyciel> <Liczba zajęć> <Minimalna liczba dni> 
	   <liczba studentów><Typ sali>
Sale: <Id Sala> <Pojemność sali> <Typ sali>
Program nauczania: <Id Program Nauczania> <Liczba Kursów> <Ids Kursy>
Ograniczenia: <Id Kurs> <Dzień> <Okres>
\end{verbatim}
Rozszerzenie formatu danych wejściowych:
\begin{enumerate}
\item Wprowadzono \verb#<Typ sali>#, który przyjmuje poniżej zdefiniowane wartości:
\begin{verbatim}
n - zwykła sala lekcyjna
w - sala warsztatowa
l - biblioteka
o - poza szkołą
e - sala gimnastyczna / boisko szkolne / basen 
c - sala laboratoryjna (komputerowa)
\end{verbatim}
\item Został wprowadzony \verb#<Typ sali># dla każdego z kursów określający w jakich rodzajach sal mogą odbywać się zajęcia z tego kursu.
\end{enumerate}